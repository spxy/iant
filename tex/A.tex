\appendix
\chapter{Lemmas}

This appendix presents some interesting results in the form of lemmas.
These lemmas are used in some of the solutions.


\begin{lemma}
  \applabel{gcdsmallestlincomb}
  The greatest common divisor (gcd) of \( a \) and \( b \) is the
  smallest positive integer that can be written as \( ax + by \) where
  \( a \), \( b \), \( x \), and \( y \) are integers such that either
  \( a \ne 0 \) or \( b \ne 0 \).
\end{lemma}

\begin{proof}
  Let \( d = (a, b) \). From the properties of gcd we know that \( d
  \ge 0 \) and in fact \( d = 0 \) if and only if \( a = b = 0 \).
  Since we have either \( a \) or \( b \) as nonzero, \( d > 0 \).
  From the properties of gcd, we know that \( d \) can be written as
  \( ax + by \). We will now show that \( d \) is the smallest such
  integer that can be written as \( ax + by \).

  Assume there exists an integer \( d' \) such that \( 0 < d' < d \)
  and \( d' = ax + by \) for some integers \( x \) and \( y \). Then
  \( d \mid d' \) by the linearity property of divisibility. Since \(
  d \mid d' \) and \( d' \ne 0 \), we get \( \lvert d \rvert \le
  \lvert d' \rvert \) by the comparison property of divisibility.
  Since \( d \ge 0 \) and \( d' > 0 \), the previous inequality is
  equivalent to \( d \le d' \). This contradicts our assumption that
  \( d' < d \).
\end{proof}


\begin{lemma}
  \applabel{multiprodcoprime}
  If \( (a, b_1) = (a, b_2) = \dots = (a, b_n) = 1 \), then \( (a, b_1
  b_2 \cdots b_n) = 1 \) where \( a \), \( b_1 \), \( b_2 \), \( \dots
  \), \( b_n \) are integers.
\end{lemma}

\begin{proof}
  We use induction on \( n \). If \( n = 2 \), the lemma is true by
  \exref{1}{2}{}. Assume that the lemma is true for \( n - 1 \).
  Therefore \( (a, b_1 b_2 \cdots b_{n - 1}) = 1 \). Since \( (a, b_1
  b_2 \cdots b_{n - 1}) = (a, b_n) = 1 \), by \exref{1}{2}{} we get \(
  (a, b_1 b_2 \cdots b_n ) = 1 \).
\end{proof}


\begin{lemma}
  \applabel{coprimeproddiv}
  If \( (a, b) = 1 \), \( a \mid c \), and \( b \mid c \), then \( ab
  \mid c \) where \( a \), \( b \), and \( c \) are integers.
\end{lemma}

\begin{proof}
  Since \( a \mid c \), we have \( c = ak \) for some integer \( k \).
  Since \( b \mid c \), we have \( b \mid ak \). Since \( b \mid ak \)
  and \( (b, a) = 1 \), by Euclid's lemma we get \( b \mid k \). Using
  the multiplication property of divisibility, we get \( ab \mid ak
  \), i.e., \( ab \mid c \).
\end{proof}


\begin{lemma}
  \applabel{multicoprimeproddiv}
  If \( a_1, a_2, \dots, a_n \) are relatively prime in pairs, \( a_1
  \mid c \), \( a_2 \mid c \), \( \dots \), \( a_n \mid c \), then \(
  a_1 a_2 \cdots a_n \mid c \) where \( a_1, a_2, \dots, a_n \), and
  \( c \) are integers.
\end{lemma}

\begin{proof}
  We use induction on \( n \). If \( n = 2 \), this lemma is true by
  \appref{coprimeproddiv}. Let \( A = a_1 a_2 \cdots a_{n-1} \).
  Assume that this lemma is true for \( n - 1 \). Therefore \( A \mid
  c \). Since \( (a_n, a_1) = (a_n, a_2) = \dots = (a_n, a_{n-1}) = 1
  \), using \appref{multiprodcoprime} we get \( (A, a_n) = 1 \). Since
  \( (A, a_n) = 1 \), \( A \mid c \), and \( a_n \mid c \), using the
  previous lemma we get \( A a_n \mid c \).
\end{proof}


\begin{lemma}
  \applabel{multicoprimecongruence}
  Let \( m_1, m_2, \dots, m_r \) are positive integers, relatively
  prime in pairs. If
  \begin{align*}
    x & \equiv a \pmod{m_1}, \\
    x & \equiv a \pmod{m_2}, \\
      & \vdotswithin{\equiv} \\
    x & \equiv a \pmod{m_r}, \\
  \end{align*}
  then \( x \equiv a \pmod{m_1 m_2 \cdots m_r} \).
\end{lemma}

\begin{proof}
  We have \( m_1 \mid (x - a) \), \( m_2 \mid (x - a) \), \( \dots \),
  \( m_r \mid (x - a) \), where \( m_1, m_2, \dots, m_r \) are
  relatively prime in pairs. Therefore, by the previous lemma, we get
  \( m_1 m_2 \cdots m_r \mid (x - a) \), i.e., \( x \equiv a \pmod{m_1
    m_2 \cdots m_r} \).
\end{proof}


\begin{lemma}
  \applabel{ndivfactorial}
  If \( n \) is composite and \( n > 4 \), then \( n \mid (n - 1)! \).
\end{lemma}

\begin{proof}
  If \( n \) is composite either \( n \) is a square of a prime or it
  isn't. If \( n = p^2 \) where \( p \) is prime, since \( n > 4 \),
  we get \( p > 2 \). Thus \( p^2 - 2p - 1 = 2 \ge 3^2 - 2 \cdot 3 - 1
  > 0 \), so we get \( p^2 - 1 > 2p \). Therefore \( (n - 1)! = (p^2 -
  1)! = 1 \cdot 2 \cdot 3 \cdot 4 \cdot 5 \cdots p \cdots 2p \cdots
  (2p - 1) \cdots (p^2 - 1) \). Thus \( p^2 \mid (n - 1)! \) or
  equivalently \( n \mid (n - 1)! \).

  If \( n \ne p^2 \) for all primes \( p \), then \( n = cd \) for
  some integers \( c \) and \( d \) such that \( 1 < c < d < n \).
  Thus \( (n - 1)! = 1 \cdot 2 \cdots c \cdots d \cdots (n - 1) \).
  Thus \( cd \mid n \) or equivalently \( n \mid (n - 1)! \).
\end{proof}
