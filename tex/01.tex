In these exercises lower case latin letters \( a \), \( b \), \( c \),
\dots, \( x \), \( y \), \( z \) represent integers. Prove each of the
statements in Exercises 1.1 through 1.6.

\begin{exercise}{1.1}
  If \( (a, b) = 1 \) and if \( c \mid a \) and \( d \mid b \), then
  \( (c, d) = 1 \).
\end{exercise}

\begin{example}
  Let \( a = 6 \) and \( b = 35 \), so we have \( (a, b) = (6, 35) = 1
  \). Let \( c = 2 \) and \( d = 5 \). We see that \( 2 \mid 6 \) and
  \( 5 \mid 35 \) and indeed \( (2, 5) = 1 \).
\end{example}

\begin{example}[Another example]
  Let \( a = 0 \) and \( b = 1 \). Let \( c = 2 \) and \( d = 1 \). We
  see that \( c \mid a \) and \( d \mid b \) and indeed \( (c, d) = 1
  \).
\end{example}

\begin{proof}
  Let \( (c, d) = g \). Since \( g \mid c \) and \( c \mid a \), by
  the transitive property of divisibility in Theorem 1.1 (b), \( g
  \mid a \). Similarly, \( g \mid b \). By the definition of gcd in
  section 1.3, we know that every common divisor of \( a \) and \( b
  \) divides \( (a, b) \), therefore \( g \mid 1 \). Since the only
  nonnegative integer that divides \( 1 \) is \( 1 \) itself, \( g = 1
  \).
\end{proof}

\begin{proof}[Another proof]
  Since \( (a, b) = 1 \), by Theorem 1.2 and the definition of gcd in
  section 1.3, we know that there are integers \( x \) and \( y \)
  such that \( ax + by = 1 \). Since \( c \mid a \) and \( d \mid b
  \), we have \( a = cm \) and \( b = dn \). Therefore \( c(mx) +
  d(ny) = (cm)x + (dn)y = ax + by = 1 \). This implies \( (c, d) = 1
  \).
\end{proof}


\begin{exercise}{1.2}
  If \( (a, b) = (a, c) = 1 \), then \( (a, bc) = 1 \).
\end{exercise}

\begin{example}
  Let \( a = 35 \), \( b = 6 \), and \( c = 14 \), so we have \( (a,
  b) = (35, 6) = 1 \) and \( (35, 14) = 1 \). Indeed \( (a, bc) = (35,
  84) = 1 \).
\end{example}

\begin{proof}
  We assume \( (a, bc) > 1 \) and obtain a contradiction. Since \( (a,
  bc) > 1 \), from the fundamental theorem of arithmetic show in
  Theorem 1.10, we know that there is a prime \( p \) such that \( p
  \mid (a, bc) \). By the definition of gcd in section 1.3, we know
  that \( p \mid a \) and \( p \mid bc \). If \( p \mid bc \), from
  Thereom 1.9 we know that \( p \mid b \) or \( p \mid c \). If \( p
  \mid a \) and \( p \mid b \), from the definition of gcd we know
  that \( p \mid (a, b) \). But \( (a, b) = 1 \). Since the only
  nonnegative integer that divides \( 1 \) is \( 1 \) itself, we have
  obtained a contradiction. If \( p \mid a \) and \( p \mid c \), we
  obtain a contradiction similarly. Thus \( (a, bc) = 1 \).
\end{proof}

\begin{proof}[Another proof]
  This proof avoids the use of the fundamental theorem of arithmetic.
  Let \( d = (a, bc) \). We will first show that \( (d, b) = 1 \) and
  then conclude later that \( d = 1 \). Let \( (d, b) = e \). Since \(
  e \mid d \) and \( d \mid a \), by the transitive property of
  divisibility in Theorem 1.1 (b), we get \( e \mid a \). Since \( e
  \mid a \) and \( e \mid b \), by the definition of gcd and Theorem
  1.3 (c), we get \( e \mid (a, b) = 1 \). Since \( e \ge 0 \) by the
  definition of gcd and since the only nonnegative integer that
  divides \( 1 \) is \( 1 \) itself, \( e = 1 \), i.e., \( (d, b) = 1
  \). Since \( d \mid bc \) and \( (d, b) = 1 \), by Thereom 1.5, we
  get \( d \mid c \). Since \( d \mid a \) and \( d \mid c \), from
  the definition of gcd, we get \( d \mid (a, c) = 1 \). Thus \( d = 1
  \).
\end{proof}

\begin{proof}[Yet another proof]
  This is a simple proof that depends only on the propeties of gcd
  shown in section 1.3. Since \( (a, b) = 1 \) and \( (a, c) = 1 \),
  there exist integers \( x_1 \), \( y_1 \), \( x_2 \), and \( y_2 \)
  such that
  \begin{align*}
    a x_1 + b y_1 & = 1, \\
    a x_2 + c y_2 & = 1.
  \end{align*}
  Therefore
  \(
    (a x_1 + b y_1) (a x_2 + c y_2) = 1 \\
    \iff a (a x_1 x_2 + c x_1 y_2 + b y_1 x_2) + bc (y_1 y_2) = 1.
  \)
  Therefore (a, bc) = 1.
\end{proof}


\begin{exercise}{1.3}
  If \( (a, b) = 1 \), then \( (a^n, b^k) = 1 \) for all \( n \le 1
  \), \( k \le 1 \).
\end{exercise}

\begin{proof}
  We assume \( (a^n, b^n) > 1 \) and obtain a contradiction. Since \(
  (a^n, b^n) > 1 \), from the fundamental theorem of arithmetic shown
  in Theorem 1.10, we know that there is a prime \( p \) such that \(
  p \mid (a^n, b^n) \). Since \( p \mid a^n \), from Theorem 1.9 we
  know that \( p \mid a \). Similarly, we know that \( p \mid b \).
  Since \( p \mid a \) and \( p \mid b \), from the definition of gcd
  in section 1.3, we get \( p \mid (a, b) = 1 \). This is a
  contradiction because the only nonnegative integer that divides \( 1
  \) is \( 1 \) itself.
\end{proof}
