% Exercise 1
\begin{exercise}{1}
  If \( (a, b) = 1 \) and if \( c \mid a \) and \( d \mid b \), then
  \( (c, d) = 1 \).
\end{exercise}

\begin{example}
  Let \( a = 6 \) and \( b = 35 \), so we have \( (a, b) = (6, 35) = 1
  \). Let \( c = 2 \) and \( d = 5 \). We see that \( 2 \mid 6 \) and
  \( 5 \mid 35 \) and indeed \( (2, 5) = 1 \).
\end{example}

\begin{example}
  Let \( a = 0 \) and \( b = 1 \). Let \( c = 2 \) and \( d = 1 \). We
  see that \( c \mid a \) and \( d \mid b \) and indeed \( (c, d) = 1
  \).
\end{example}

\begin{proof}
  Let \( (c, d) = g \). Since \( g \mid c \) and \( c \mid a \), by
  the transitive property of divisibility in Theorem 1.1 (b), \( g
  \mid a \). Similarly, \( g \mid b \). By the definition of gcd, and
  the following definition, we know that every common divisor of \( a
  \) and \( b \) divides \( (a, b) \), therefore \( g \mid 1 \). Since
  the only integer that divides \( 1 \) is \( 1 \) itself, \( g = 1
  \).
\end{proof}

\begin{proof}
  Since \( (a, b) = 1 \), by Theorem 1.2 and the definition of gcd in
  section 1.3, we know that there are integers \( x \) and \( y \)
  such that \( ax + by = 1 \). Since \( c \mid a \) and \( d \mid b
  \), we have \( a = cm \) and \( b = dn \). Therefore \( c(mx) +
  d(ny) = (cm)x + (dn)y = ax + by = 1 \). This implies \( (c, d) = 1
  \).
\end{proof}


% Exercise 2
\begin{exercise}{2}
  If \( (a, b) = (a, c) = 1 \), then \( (a, bc) = 1 \).
\end{exercise}

\begin{example}
  Let \( a = 35 \), \( b = 6 \), and \( c = 14 \), so we have \( (a,
  b) = (35, 6) = 1 \) and \( (35, 14) = 1 \). Indeed \( (a, bc) = (35,
  84) = 1 \).
\end{example}

\begin{proof}
  Let \( d = (a, bc) \). We will now show that \( (d, b) = 1 \). Let
  \( (d, b) = e \). Since \( e \mid d \) and \( d \mid a \), by the
  transitive property of divisibility in Theorem 1.1 (b), we get \( e
  \mid a \). Since \( e \mid a \) and \( e \mid b \), by the
  definition of gcd and Theorem 1.3 (c), we get \( e \mid (a, b) = 1
  \). Since \( e \ge 0 \) by the definition of gcd and since the only
  nonnegative integer that divides \( 1 \) is \( 1 \) itself, \( e = 1
  \), i.e., \( (d, b) = 1 \).
\end{proof}

