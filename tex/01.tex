\setchapter{1}{The Fundamental Theorem of Arithmetic}


In these exercises lower case latin letters \( a \), \( b \), \( c \),
\dots, \( x \), \( y \), \( z \) represent integers. Prove each of the
statements in Exercises 1.1 through 1.6.


\begin{exercise}{1}{}
  If \( (a, b) = 1 \) and if \( c \mid a \) and \( d \mid b \), then
  \( (c, d) = 1 \).
\end{exercise}

\begin{example}
  Let \( a = 6 \) and \( b = 35 \), so we have \( (a, b) = (6, 35) = 1
  \). Let \( c = 2 \) and \( d = 5 \). We see that \( 2 \mid 6 \) and
  \( 5 \mid 35 \) and indeed \( (2, 5) = 1 \).
\end{example}

\begin{example}[Another example]
  Let \( a = 0 \) and \( b = 1 \). Let \( c = 2 \) and \( d = 1 \). We
  see that \( c \mid a \) and \( d \mid b \) and indeed \( (c, d) = 1
  \).
\end{example}

\begin{proof}
  Let \( (c, d) = g \). Since \( g \mid c \) and \( c \mid a \), by
  the transitive property of divisibility in Theorem 1.1 (b), \( g
  \mid a \). Similarly, \( g \mid b \). By the definition of gcd in
  section 1.3, we know that every common divisor of \( a \) and \( b
  \) divides \( (a, b) \), therefore \( g \mid 1 \). Since the only
  nonnegative integer that divides \( 1 \) is \( 1 \) itself, \( g = 1
  \).
\end{proof}

\begin{proof}[Another proof]
  Since \( (a, b) = 1 \), by Theorem 1.2 and the definition of gcd in
  section 1.3, we know that there are integers \( x \) and \( y \)
  such that \( ax + by = 1 \). Since \( c \mid a \) and \( d \mid b
  \), we have \( a = cm \) and \( b = dn \). Therefore \( c(mx) +
  d(ny) = (cm)x + (dn)y = ax + by = 1 \). This implies \( (c, d) = 1
  \).
\end{proof}


\begin{exercise}{2}{}
  If \( (a, b) = (a, c) = 1 \), then \( (a, bc) = 1 \).
\end{exercise}

\begin{example}
  Let \( a = 35 \), \( b = 6 \), and \( c = 14 \), so we have \( (a,
  b) = (35, 6) = 1 \) and \( (35, 14) = 1 \). Indeed \( (a, bc) = (35,
  84) = 1 \).
\end{example}

\begin{proof}
  We assume \( (a, bc) > 1 \) and obtain a contradiction. Since \( (a,
  bc) > 1 \), from the fundamental theorem of arithmetic show in
  Theorem 1.10, we know that there is a prime \( p \) such that \( p
  \mid (a, bc) \). By the definition of gcd in section 1.3, we know
  that \( p \mid a \) and \( p \mid bc \). If \( p \mid bc \), from
  Thereom 1.9 we know that \( p \mid b \) or \( p \mid c \). If \( p
  \mid a \) and \( p \mid b \), from the definition of gcd we know
  that \( p \mid (a, b) \). But \( (a, b) = 1 \). Since the only
  nonnegative integer that divides \( 1 \) is \( 1 \) itself, we have
  obtained a contradiction. If \( p \mid a \) and \( p \mid c \), we
  obtain a contradiction similarly. Thus \( (a, bc) = 1 \).
\end{proof}

\begin{proof}[Another proof]
  This proof avoids the use of the fundamental theorem of arithmetic.
  Let \( d = (a, bc) \). We will first show that \( (d, b) = 1 \) and
  then conclude later that \( d = 1 \). Let \( (d, b) = e \). Since \(
  e \mid d \) and \( d \mid a \), by the transitive property of
  divisibility in Theorem 1.1 (b), we get \( e \mid a \). Since \( e
  \mid a \) and \( e \mid b \), by the definition of gcd and Theorem
  1.3 (c), we get \( e \mid (a, b) = 1 \). Since \( e \ge 0 \) by the
  definition of gcd and since the only nonnegative integer that
  divides \( 1 \) is \( 1 \) itself, \( e = 1 \), i.e., \( (d, b) = 1
  \). Since \( d \mid bc \) and \( (d, b) = 1 \), by Thereom 1.5, we
  get \( d \mid c \). Since \( d \mid a \) and \( d \mid c \), from
  the definition of gcd, we get \( d \mid (a, c) = 1 \). Thus \( d = 1
  \).
\end{proof}

\begin{proof}[Yet another proof]
  This is a simple proof that depends only on the propeties of gcd
  shown in section 1.3. Since \( (a, b) = 1 \) and \( (a, c) = 1 \),
  there exist integers \( x_1 \), \( y_1 \), \( x_2 \), and \( y_2 \)
  such that
  \begin{align*}
    a x_1 + b y_1 & = 1, \\
    a x_2 + c y_2 & = 1.
  \end{align*}
  Therefore
  \(
    (a x_1 + b y_1) (a x_2 + c y_2) = 1 \\
    \iff a (a x_1 x_2 + c x_1 y_2 + b y_1 x_2) + bc (y_1 y_2) = 1.
  \)
  Therefore (a, bc) = 1.
\end{proof}


\begin{exercise}{3}{}
  If \( (a, b) = 1 \), then \( (a^n, b^k) = 1 \) for all \( n \le 1
  \), \( k \le 1 \).
\end{exercise}

\begin{proof}
  We assume \( (a^n, b^n) > 1 \) and obtain a contradiction. Since \(
  (a^n, b^n) > 1 \), from the fundamental theorem of arithmetic shown
  in Theorem 1.10, we know that there is a prime \( p \) such that \(
  p \mid (a^n, b^n) \). Since \( p \mid a^n \), from Theorem 1.9 we
  know that \( p \mid a \). Similarly, we know that \( p \mid b \).
  Since \( p \mid a \) and \( p \mid b \), from the definition of gcd
  in section 1.3, we get \( p \mid (a, b) = 1 \). This is a
  contradiction because the only nonnegative integer that divides \( 1
  \) is \( 1 \) itself.
\end{proof}


\begin{exercise}{4}{}
  If \( (a, b) = 1 \), then \( (a + b, a - b) \) is either \( 1 \) or
  \( 2 \).
\end{exercise}

\begin{proof}
  Let \( d = (a + b, a - b) \). If \( a + b \) and \( a - b \) are
  relatively prime, then \( d = 1 \). If they are not relatively
  prime, then \( d > 1 \). Then by the fundamental theorem of
  arithmetic shown in Theorem 1.10, there is a prime \( p \) such that
  \( p \mid d \). From the definition of gcd in section 1.3, we get \(
  p \mid (a + b) \) and \( p \mid (a - b) \). From the linearity
  property of divisibility in Theorem 1.1 (c), we get \( p \mid 2a \)
  and \( p \mid 2b \). Now there are two cases to consider: \( p \mid
  2 \) and \( p \nmid 2 \). If \( p \mid 2 \), \( p = 2 \) because the
  only prime that divides \( 2 \) is \( 2 \) itself. If \( p \nmid 2
  \), by Theorem 1.8, we have \( (p, 2) = 1 \). If \( p \mid 2a \) and
  \( (p, 2) = 1 \), then by Euclid's lemma shown in Theorem 1.5, we
  have \( p \mid a \). We can similarly show that if \( p \nmid 2 \),
  then \( p \mid b \). Thus by the definition of gcd, we get \( p \mid
  (a, b) = 1 \). Since the only positive integer that divides \( 1 \)
  is \( 1 \) itself, we have obtained a contradiction. Therefore, if
  \( d > 1 \), the only prime \( p \) such that \( p \mid d \) is \( p
  = 2 \). Thus \( d = 2 \). We have shown that \( d = 1 \) or \( d = 2
  \).
\end{proof}

\begin{proof}[Another proof]
  This is a simpler proof that depends only on the properties of gcd.
  Since \( (a, b) = 1 \), from the definition of gcd in section 1.3,
  we know that there are integers \( x \) and \( y \) such that \( ax
  + by = 1 \). Therefore,
  \[ (a + b)(x + y) + (a - b)(x - y) = 2(ax + by) = 2. \]
  Thus from the linearity property in Theorem 1.1 (c), we know that \(
  (a + b, a - b) \mid 2 \). Now from the comparison property in
  Theorem 1.1 (i), we know that \( (a + b, a - b) \le 2 \).
\end{proof}


\begin{exercise}{5}{}
  If \( (a, b) = 1 \), then \( (a + b, a^2 - ab + b^2) \) is either \(
  1 \) or \( 3 \).
\end{exercise}

\begin{proof}
  Let \( d = (a + b, a^2 - ab + b^2) \). If \( (a + b) \) and \( a^2 -
  ab + b^2 \) are relatively prime, then \( d = 1 \). If they are not
  relatively prime, then \( d > 1 \). Then by the fundamental theorem
  of arithmetic in Theorem 1.10, there is a prime \( p \) such that \(
  p \mid d \). From the definition of gcd in section 1.3, we get \( p
  \mid (a + b) \) and \( p \mid (a^2 - ab + b^2) \). From the
  linearity property of divisibility in Theorem 1.1 (c), we get \( p
  \mid (a + b)^2 - (a^2 - ab + b^2) = 3ab \). Thus from Theorem 1.9,
  \( p \mid 3 \) or \( p \mid a \) or \( p \mid b \). If \( p \mid a
  \), since \( p \mid (a + b) \), using the linearity property of
  divisibility again, we see that \( p \mid b \). But then by
  properties of gcd, \( p \mid (a, b) = 1 \). This is a contradiction,
  since the only positive integer that divides \( 1 \) is \( 1 \)
  itself. Therefore we conclude that \( p \nmid a \). We can show
  similarly that \( p \nmid b \). Thus we are left with only \( p \mid
  3 \). Using the contrapositive of Theorem 1.9, we conclude that \( p
  \nmid ab \). Thus using Theorem 1.8, we conclude that \( (p, ab) = 1
  \). Since \( p \mid 3ab \), using Euclid's Lemma in Theorem 1.5, we
  get \( p \mid 3 \). Thus \( p = 3 \). Therefore if \( d > 1 \), the
  only prime \( p \) such that \( p \mid d \) is \( p = 3 \). We have
  shownt that \( d = 1 \) or \( d = 3 \).
\end{proof}


\begin{exercise}{6}{}
  If \( (a, b) = 1 \) and \( d \mid (a + b) \), then \( (a, d) = (b, d) = 1 \).
\end{exercise}

\begin{example}
  Let \( a = 5 \) and \( b = 7 \). Thus \( a + b = 12 \). We see that
  \( 2 \mid 12 \) and indeed \( (2, 5) = (2, 7) = 1 \). We can pick
  any other divisor of \( d \) of \( 12 \) and indeed \( (d, 5) = (d,
  7) = 1 \) holds.
\end{example}

\begin{proof}
  Let \( g = (a, d) \). By the definition of gcd in section 1.3, \( g
  \mid a \) and \( g \mid d \). Since \( g \mid d \) and \( d \mid (a
  + b) \), using the transitive property of divisiblity in Theorem 1.1
  (b), we get \( g \mid (a + b) \). Since \( g \mid a \) and \( g \mid
  (a + b) \), using the linearity property of divisibility in Theorem
  1.1 (c), we get \( g \mid b \). Since \( g \mid a \) and \( g \mid b
  \), using the property of gcd, we get \( g \mid (a, b) = 1 \). But
  the only nonnegative integer that divides \( 1 \) is \( 1 \) itself,
  therefore, \( g = 1 \). Therefore, \( (a, d) = 1 \). We can
  similarly show that \( (b, d) = 1 \).
\end{proof}


\begin{exercise}{7}{}
  A rational number \( a/b \) with \( (a, b) = 1 \) is called a
  \emph{reduced fraction}. If the sum of two reduced fractions in an
  integer, say \( (a/b) + (c/d) = n \), prove that \( \lvert -b \rvert
  = \lvert d \rvert \).
\end{exercise}

\begin{proof}
  Since \( (a/b) + (c/d) = n \), we get \( ad + bc = nbd \). Thus \(
  ad = b(nd - c) \). This shows that \( b \mid ad \). Since \( b \mid
  ad \) and \( (a, b) = 1 \), from Euclid's lemma in Theorem 1.5 we
  get, \( b \mid d \). We can similarly show that \( d \mid bc \) and
  thus \( d \mid b \). Since \( b \mid d \) and \( d \mid b \), from
  Theorem 1.1 (i), we get \( \lvert d \rvert = \lvert b \rvert \).
\end{proof}


\begin{exercise}{8}{}
  An integer is called \emph{squarefree} if it is not divisible by the
  square of any prime. Prove that for every \( n \ge 1 \) there exist
  uniquely determined \( a > 0 \) and \( b > 0 \) such that \( n = a^2
  b \), where \( b \) is squarefree.
\end{exercise}

\begin{proof}
  TODO
\end{proof}
