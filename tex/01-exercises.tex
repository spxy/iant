% Exercise 1
\begin{exercise}{1}
  If \( (a, b) = 1 \) and if \( c \mid a \) and \( d \mid b \), then
  \( (c, d) = 1 \).
\end{exercise}

\begin{example}
  Let \( a = 6 \) and \( b = 35 \). Let \( c = 2 \) and \( d = 5 \). We
  see that \( c \mid a \) and \( d \mid b \) and indeed \( (c, d) = 1
  \).
\end{example}

\begin{example}
  Let \( a = 0 \) and \( b = 1 \). Let \( c = 2 \) and \( d = 1 \). We
  see that \( c \mid a \) and \( d \mid b \) and indeed \( (c, d) = 1 \).
\end{example}

\begin{proof}
  Let \( (c, d) = g \). Since \( g \mid c \) and \( c \mid a \), by
  Theorem 1.1 (b) (transitive property), \( g \mid a \). Similarly, \( g
  \mid b \). By Theorem 1.3 (c) and the following definition, we know
  that every common divisor of \( a \) and \( b \) divides \( (a, b) \),
  therefore \( g \mid 1 \).

  Since the only integer that divides \( 1 \) is \( 1 \) itself, \( g =
  1 \).
\end{proof}
