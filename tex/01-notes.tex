\begin{notes}[\S 1.1 The principle of induction]
Imagine you have a friend Alice standing in front of a table with
infinite number of apples. The apples are labelled \( a_1 \), \( a_2 \),
\dots. Alice has a basket that we will denote as \( Q \). You ask Alice
to play a game with these two rules:

\begin{enumerate}
  \item Alice must pick apple \( a_1 \) and put it in her basket \( Q \).
  \item If Alice picks apple \( a_n \) and puts it in her basket, she
    must also pick apple \( a_{n + 1} \) and put it in her basket.
\end{enumerate}

What happens in the end? If Alice follows both the rules, she will be
busy picking all infinite apples and putting them in her basket forever.
She must pick apple \( a_1 \) as per rule 1. Now since she has picked
apple \( a_1 \), rule 2 requires that she must pick apple \( a_2 \) too
and put it in her basket. Rule 2 applies again because she has picked
apple \( a_2 \), so she must now pick apple \( a_3 \) too and put it in
her basket. The basket would eventually contain all infinite apples.
\end{notes}

\begin{notes}[\S 1.1 The well-ordering principle]
See
\url{https://brilliant.org/wiki/the-well-ordering-principle/#equivalence-with-induction}
for a proof that this is equivalent to the principle of induction.  
\end{notes}
